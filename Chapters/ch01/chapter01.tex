\chapter{The Beginning of Googology} % or maybe "origin of googology"?

\section{The Simplest Unary Operation}

\subsection{The Positive Integers}
The set of positive integers, $\mathbb{N}^*$, is defined as satisfying the following conditions:
\begin{enumerate}
	\item For any arbitrary positive integer $a$, its successor, $a^+$ is a positive integer as well.
	\item There exists a positive integer such that no positive integer's successor is that integer. Notate this as $1$.
	\item Except for $1$, every positive integer is the successor of a positive integer.
	\item If $1 \in S$, and whenever $n \in S$, $n^+ \in S$ as well; then $S = \mathbb{N}^*$.
\end{enumerate}

But problems arise. Until now, we might have wondered how to express things like ``1's successor," ``1's successor's successor," and so on.
Could they be expressed as $1^+$, $1^{++}$, etc?

For a long time, people defined [TODO] as follows: $1^+=2, 2^+=3, 3^+=4, 4^+=5, 5^+=6, 6^+=7, 7^+=8, 8^+=9, 9^+=10.$ 
We use two digits in ``10" to refer to the successor of the number 9.

Then, $10^+=11, 11^+=12, \ldots, 19^+=20, \ldots, 99^+=100, \ldots$
Thus, we now have a way to theoretically express any positive integer.

However, this simple operation by itself is practically useless. The really useful stuff is the following...

\subsection{Addition}
Addition is defined as follows, expressed using the symbol ``+".
	$$ a+1=a^+ $$
	$$ a+b^+=(a+b)^+ $$
very simple indeed! But this definition is too abstract, in fact we can define it like this:
	$$ a+b=a^{++++\ldots+} $$
(with $b$ ``+"es). This new definition is clearer.

However, sometimes addition isn't enough. Then we need...

\subsection{Multiplication}
Multiplication is defined as follows, expressed using the symbol ``$\times$".
	$$ a \times 1 = a $$
	$$ a \times b^+ = a \times b + a $$
This is also a very simple definition. A bit more accessible definition:
        $$ a \times b = a + a + a + \ldots + a $$ (with ``b" many ``a"s.)
From this, one can find that $10\times 10 = 100, 100\times 10=1000, 100\ldots0$ (with $n$ zeroes)
$\times 10 = 100\ldots0$ (with $n+1$ zeroes.) This leads us into another operation:

\subsection{Exponentiation}
Exponentiation is usually denoted by the symbol ``$\textrm{\^{}}$", and defined as:
	$$ a\textrm{\^{}} 1=a $$
	$$ a\textrm{\^{}} b^+ =a\textrm{\^{}}b \times a $$

Exponentiation is neither commutative nor associative. Usually, it is right-associative, so $ a\textrm{\^{}}b\textrm{\^{}}c $ 
gets interpreted as $ a\textrm{\^{}}(b\textrm{\^{}}c). $

Sometimes, we don't use any symbol to express exponentiation, instead choosing to write $a^b$ where the $b$ being on top signifies exponentiation.
Then, $a^b=a\textrm{\^{}}b.$

A more intuitive definition is: $a^b=a\times a\times\ldots\times a $ (with $b$ ``$a$"s). However, higher level operaations were not developed 
for a long time, until recently. People used to "create" many numbers, name them, and use them to express any positive integer.
