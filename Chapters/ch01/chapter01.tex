\chapter{The Beginning of Googology} % or maybe "origin of googology"?

\section{The Simplest Unary Operation}

\subsection{The Positive Integers}
The set of positive integers, $\mathbb{N}^*$, is defined as satisfying the following conditions:
\begin{enumerate}
	\item For any arbitrary positive integer $a$, its successor, $a^+$ is a positive integer as well.
	\item There exists a positive integer such that no positive integer's successor is that integer. Notate this as $1$.
	\item Except for $1$, every positive integer is the successor of a positive integer.
	\item If $1 \in S$, and whenever $n \in S$, $n^+ \in S$ as well; then $S = \mathbb{N}^*$.
\end{enumerate}

But problems arise. Until now, we might have wondered how to express things like ``1's successor," ``1's successor's successor," and so on.
Could they be expressed as $1^+$, $1^{++}$, etc?

For a long time, people defined [TODO] as follows: $1^+=2, 2^+=3, 3^+=4, 4^+=5, 5^+=6, 6^+=7, 7^+=8, 8^+=9, 9^+=10.$ 
We use two digits in ``10" to refer to the successor of the number 9.

Then, $10^+=11, 11^+=12, \ldots, 19^+=20, \ldots, 99^+=100, \ldots$
Thus, we now have a way to theoretically express any positive integer.

However, this simple operation by itself is practically useless. The really useful stuff is the following...

\subsection{Addition}

